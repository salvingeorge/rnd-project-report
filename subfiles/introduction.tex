%!TEX root = ../report.tex
\documentclass[../report.tex]{subfiles}

\begin{document}
    \section{Introduction}
    \label{sec:introduction}

    % This is a template for MAS R\&D projects, based on \emph{IEEETran}.
    % Here are some preliminaries about some common things you need to do to use the template:
    % \begin{itemize}
    %     \item Add your references to the file \emph{references.bib} and cite them as Mustermann and Smith \cite{referenceexample} (if there are more than three authors, cite as Mustermann et al. \cite{referenceexample}).
    %     \item Refer to sections as Sec. \ref{sec:introduction}.
    %     \item You can include figures as follows (note that the figure caption is below the figure).
    %     \begin{figure}[ht]
    %         \centering
    %         \includegraphics[width=0.8\linewidth]{figures/b-it-logo.pdf}
    %         \caption{My caption}
    %         \label{fig:figureexample}
    %     \end{figure}
    %     Refer to figures as Fig. \ref{fig:figureexample}.
    %     \item You can add tables as follows (note that the table caption is above the table).
        % \begin{table}[ht]
        %     \caption{My caption}
        %     \label{tab:tableexample}
        %     \begin{tabular}{M{0.45\linewidth} M{0.45\linewidth}}
        %         \hline
        %         \cellcolor{gray!10!white} Header 1 & \cellcolor{gray!10!white} Header 2 \\\hline
        %         Cell 1 & Cell 2 \\\hline
        %         Cell 3 & Cell 4 \\\hline
        %     \end{tabular}
        % \end{table}
    %     Refer to tables as Tab. \ref{tab:tableexample}.
    %     \item You can add equations as follows.
    %     \begin{equation}
    %         f(x) = \frac{1}{\sigma\sqrt{2\pi}}e^{-\frac{1}{2}\left( \frac{x - \mu}{\sigma} \right)^2}
    %         \label{eq:equationexample}
    %     \end{equation}
    %     Refer to equations as Eq. \ref{eq:equationexample}.
    % \end{itemize}
    It is claimed that life would not be possible without radioactivity, which is an essential component of life. Life on Earth began when naturally occurring radioactivity heated
    the Earth's core \cite{DOEExplainsRadioactivity}. The science of radioactivity has come a long way since Henri Becquerel discovered it in 1896, and it is now a prerequisite for
    many modern technologies. Radioactive elements are used in scientific studies to determine the age of rocks and fossils. For example, radioactive materials are
    employed in health imaging and disease treatment. In energy, radioactive materials are utilized in nuclear power plants and spacecraft.
    % \vspace{0.5cm}


    However, the inherent properties of radioactive materials that make them valuable also present significant challenges in handling and security. These materials are
    invisible to the naked eye and can be easily concealed, making their detection and tracking a complex technological challenge. The risks associated with radioactive materials
    are compounded by their potential for malicious use, whether as weapons or as a means of environmental contamination.
    \subsection{Motivation}
    \label{sec:introduction:motivation}

    % Describe the context of your work and the motivation for it.
    It is essential to develop trustworthy techniques for locating radioactive sources. First, these methods can significantly enhance security measures by helping law enforcement and security 
    agencies spot possible threats faster. Second, quick detection of radioactive leakage is essential for preserving operational safety and safeguarding the environment and employees in industrial 
    settings such as nuclear power plants. Localizing radioactive sources is intriguing yet challenging, highlighting the need for continued research. Strong simulation frameworks are required 
    since testing with actual radioactive sources can be dangerous and complicated by nature. Furthermore, comprehending the complex physical behavior of radiation, particularly particle 
    attenuation and scattering, presents intriguing modeling problems that affect simulations' accuracy and localization's efficacy. Although various methods have been put forth, they are less 
    feasible for rapid source detection because they are computationally intensive and necessitate broad area searches.  It is very important to have a set of tools and methods that can effectively
    localize the radiation source without human interference, as the need for this is very relevant in recent times. This way, we can ensure public safety and environmental protection. 

    \subsection{Problem Statement}
    \label{sec:introduction:problem_statement}

    % Describe the problem you are addressing in the work.
    The fundamental challenge in radioactive source localization lies in using sensor measurements and prior knowledge to estimate the position and intensity of unknown radiation 
    sources within an environment. Current approaches face several key limitations: they often require an exhaustive and complete search of the entire search area, struggle with sensor noise and measurement 
    uncertainty, perform poorly in environments with limited accessibility, and suffer from high computational complexity when dealing with dynamic and large search spaces.
   
    The core research problem encompasses three main aspects. First, the development of a physics-accurate simulation framework that can reliably model radiation behavior, including particle 
    attenuation and scattering effects. Second, a comprehensive evaluation framework capable of assessing and comparing different localization approaches must be created. Third, the 
    implementation and analysis of various localization algorithms, examining their computational complexity, resource requirements, accuracy, and efficiency across different environmental conditions.
    \label{sec:introduction:proposed_approach}

    Despite advances in radiation source localization, existing methods often rely on simplified models that inadequately capture real-world radiation physics and 
    lack systematic ways to compare different approaches. Current solutions frequently compromise detection speed and computational efficiency, limiting their 
    practical application.
    
    
    Our research addresses these challenges through three key contributions: the development of a physics-accurate simulation framework incorporating realistic 
    radiation behavior, the implementation of three complementary localization algorithms (information-gain-based, rollout-based, and inverse square optimization), 
    and the establishment of a comprehensive evaluation methodology. Together, these contributions enable a detailed comparison of the implemented algorithms given 
    the chosen metrics, advancing the field toward more practical and efficient solutions.

    % Write a short summary of your proposed approach.
    We present a framework that integrates simulation, evaluation, and algorithm implementation to tackle these challenges. The simulation component uses established radiation physics models 
    to create realistic test scenarios, incorporating inverse square law, attenuation, and scattering effects. The evaluation framework provides quantitative metrics for assessing algorithm 
    performance, including computational efficiency, accuracy, and robustness to noise. We implement and compare several localization methods, including rollout algorithms, entropy-based 
    approaches, and optimization-based methods. Each method is systematically evaluated under various environmental conditions and constraints. We aim to develop an integrated system that 
    allows for robust and flexible implementation and testing of various radiation source localization algorithms.



\end{document}
