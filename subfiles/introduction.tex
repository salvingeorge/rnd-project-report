%!TEX root = ../report.tex
\documentclass[../report.tex]{subfiles}

\begin{document}
    \section{Introduction}
    \label{sec:introduction}

    % This is a template for MAS R\&D projects, based on \emph{IEEETran}.
    % Here are some preliminaries about some common things you need to do to use the template:
    % \begin{itemize}
    %     \item Add your references to the file \emph{references.bib} and cite them as Mustermann and Smith \cite{referenceexample} (if there are more than three authors, cite as Mustermann et al. \cite{referenceexample}).
    %     \item Refer to sections as Sec. \ref{sec:introduction}.
    %     \item You can include figures as follows (note that the figure caption is below the figure).
    %     \begin{figure}[ht]
    %         \centering
    %         \includegraphics[width=0.8\linewidth]{figures/b-it-logo.pdf}
    %         \caption{My caption}
    %         \label{fig:figureexample}
    %     \end{figure}
    %     Refer to figures as Fig. \ref{fig:figureexample}.
    %     \item You can add tables as follows (note that the table caption is above the table).
    %     \begin{table}[ht]
    %         \caption{My caption}
    %         \label{tab:tableexample}
    %         \begin{tabular}{M{0.45\linewidth} M{0.45\linewidth}}
    %             \hline
    %             \cellcolor{gray!10!white} Header 1 & \cellcolor{gray!10!white} Header 2 \\\hline
    %             Cell 1 & Cell 2 \\\hline
    %             Cell 3 & Cell 4 \\\hline
    %         \end{tabular}
    %     \end{table}
    %     Refer to tables as Tab. \ref{tab:tableexample}.
    %     \item You can add equations as follows.
    %     \begin{equation}
    %         f(x) = \frac{1}{\sigma\sqrt{2\pi}}e^{-\frac{1}{2}\left( \frac{x - \mu}{\sigma} \right)^2}
    %         \label{eq:equationexample}
    %     \end{equation}
    %     Refer to equations as Eq. \ref{eq:equationexample}.
    % \end{itemize}
    It is claimed that life would not be possible without radioactivity, which is an essential component of life. Life on Earth began when naturally occurring radioactivity heated
    the Earth's core \cite{DOEExplainsRadioactivity}. The science of radioactivity has come a long way since Henri Becquerel discovered it in 1896, and it is now a prerequisite for
    many modern technologies. Radioactive elements are used in scientific studies to determine the age of rocks and fossils. For example, in health, radioactive materials are
    employed in imaging and treating disease. In energy, radioactive materials are utilized in nuclear power plants and also to power spacecraft.
    % \vspace{0.5cm}


    However, the inherent properties of radioactive materials that make them valuable also present significant challenges in their handling and security. These materials are
    invisible to the naked eye and can be easily concealed, making their detection and tracking a complex technological challenge. The risks associated with radioactive materials
    are compounded by their potential for malicious use, whether as weapons or as means of environmental contamination.
    \subsection{Motivation}
    \label{sec:introduction:motivation}

    % Describe the context of your work and the motivation for it.
    The development of reliable methods for localizing radioactive sources is crucial for several reasons. First, these techniques can significantly enhance security measures by 
    helping law enforcement and security organizations identify potential threats more quickly. Second, in industrial settings such as nuclear power plants, rapid detection of 
    radioactive leaks is essential for maintaining operational safety and protecting both workers and the surrounding environment.
    The field of radioactive source localization faces several significant challenges that underscore the importance of this research. Testing with real radioactive sources is 
    inherently difficult and potentially hazardous, necessitating the development of accurate simulation frameworks. Furthermore, the physical behavior of radiation, particularly
    particle attenuation and scattering, presents complex modeling challenges that affect both the simulation accuracy and the effectiveness of localization methods. While 
    previous approaches exist, many are computationally expensive and require exhaustive area searches, making them impractical for rapid source identification.
    These challenges, combined with the increasing incidents of radioactive material mishandling, highlight the urgent need for more efficient and practical solutions. The 
    development of improved localization methods could significantly enhance our ability to manage and secure radioactive materials, ultimately contributing to both public 
    safety and environmental protection.

    \subsection{Problem Statement}
    \label{sec:introduction:problem_statement}

    % Describe the problem you are addressing in the work.
    The fundamental challenge in radioactive source localization lies in using sensor measurements and prior knowledge to estimate the position and intensity of unknown radiation 
    sources within an environment. Current approaches face several key limitations: they often require exhaustive search of the entire space, struggle with sensor noise and measurement 
    uncertainty, perform poorly in environments with limited accessibility, and suffer from high computational complexity when dealing with dynamic and large search spaces.

    The core research problem encompasses three main aspects. First, the development of a physics-accurate simulation framework that can reliably model radiation behavior, including particle 
    attenuation and scattering effects. Second, the creation of a comprehensive evaluation framework capable of assessing and comparing different localization approaches. Third, the 
    implementation and analysis of various localization algorithms, examining their computational complexity, resource requirements, accuracy, and efficiency across different environmental conditions.
    \subsection{Proposed Approach}
    \label{sec:introduction:proposed_approach}

    % Write a short summary of your proposed approach.
    To address these challenges, we propose an integrated framework that combines simulation, evaluation, and algorithm implementation. The simulation component uses established radiation physics models 
    to create realistic test scenarios, incorporating inverse square law, attenuation, and scattering effects. The evaluation framework provides quantitative metrics for assessing algorithm performance, 
    including computational efficiency, accuracy, and robustness to noise. We implement and compare several localization methods, including rollout algorithms, entropy-based approaches, and optimization-based methods. Each method is systematically evaluated under various environmental conditions and constraints. Through this comprehensive approach, we aim to advance both the theoretical understanding and 
    practical application of radioactive source localization.

\end{document}
