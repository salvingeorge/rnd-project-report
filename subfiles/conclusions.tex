%!TEX root = ../report.tex
\documentclass[../report.tex]{subfiles}

\begin{document}
    \section{Conclusions}
    \label{sec:conclusions}
    Developing and evaluating our framework for radioactive source localization has provided valuable insights into the relative performance of different 
    localization strategies. This method combines physics-based radiation simulation with systematic evaluation methods to compare three distinct approaches: information-gain-based
    and inverse square law optimization algorithms. Our experiments and evaluation across multiple source configurations demonstrate that each method can localize sources with
    varying efficiency and accuracy. The information-gain-based algorithm is a winner regarding convergence time and achieving the lowest
    position error in source localization. The algorithm while managed to achieve the lowerst position error, also experienced a few number of highest position error as visible in Figure~\ref{fig:prediction_violin_plot}. 
    The rollout algorithm also performs well but with a slightly higher convergence time and consistently predicts a position error of around 7 meters due to the selection of a grid resolution of 5 meters. While effective, the inverse square law optimization algorithm 
    takes significantly longer to converge and, with some source configurations, position in this case, witnesses a higher error. It is also worth noting that the evaluation metrics are the convergence time and the position error. The convergence time also includes the movement time, and a 
    drone with better movement control might achieve better performance in the time taken. The entropy algorithm managed to achieve the best accuracy of upto
    just around 2 meters, but the rollout-based method has managed to achieve around 5 meters due to the grid resolution. The inverse square law optimization 
    method fluctuates quite significantly due to the area coverage and source position. The higher and lowest position errors in information-gain-based method represents the tradeoff between 
    the time taken to converge and the accuracy achieved. 

    \subsection{Summary}
    \label{sec:conclusions:summary}

    Our comprehensive evaluation of three distinct approaches to radiation source localization based on source location $(30,50)$ in 
    table~\ref{tab:algorithm_comparison}- information-gain-based, rollout-based, and inverse square law 
    optimization - revealed several significant findings. Statistical analysis demonstrated that while all three methods achieved comparable accuracy in source 
    localization, with errors ranging from 4.5 to 14.4 meters, they exhibited marked differences in convergence speed. The information-gain-based approach 
    consistently demonstrated superior performance, achieving convergence in 43-126 seconds, followed by the rollout algorithm at 90-177 seconds. Both methods 
    significantly outperformed the baseline inverse square method, which required 415-470 seconds for convergence.

    The Wilcoxon Signed-Rank Test demonstrated significant differences in convergence times among the algorithms, with large effect sizes (r = 0.8864) supporting 
    this finding. This statistical evidence and practical performance metrics strongly validate the effectiveness of information-theoretic methods in radiation 
    source localization.

    \subsection{Contributions}
    \label{sec:conclusions:contributions}

    This research makes several contributions to the field of autonomous radiation source localization. First, we have developed a physics-based 
    simulation framework that incorporates realistic radiation behavior, including particle attenuation and scattering effects. This framework provides a reliable
    platform for testing and evaluating localization algorithms under controlled conditions.

    Second, our implementation and comparative analysis of three distinct localization approaches establishes quantitative benchmarks for algorithm performance. 
    The systematic evaluation methodology we developed, combining both practical performance metrics and rigorous statistical analysis, provides a framework 
    for assessing future algorithms in this domain.

    Third, our results demonstrated that, in the domain of radioactive source localization, the information-based methods could perform significantly better than
    the traditional predefined trajectory following methods without compromising accuracy and time. This study offers a promising direction for developing 
    autonomous methods that may need to find radioactive sources

    Fourth, we have developed a comprehensive evaluation framework with an intuitive graphical interface that enables systematic analysis of localization algorithms.
    This framework features built-in statistical analysis capabilities, customizable visualization tools, and an extensible architecture that facilitates easy 
    integration of new algorithms and evaluation metrics. The framework's modular design allows researchers to incorporate additional statistical tests and 
    performance metrics, enabling thorough comparative analysis tailored to specific research requirements.


    \subsection{Future Work}
    \label{sec:conclusions:future_work}

    Several promising directions exist for extending and improving this research. First, the framework could be extended to handle multiple sources, which would be 
    more challenging but more realistic. This would require extending the current methods to handle the overlapping of radiation fields. Secondly, the detection time 
    and total time for rollout algorithm could be brought down by detecting the source at an earlier stage with lower confidence. This needs further investigation as
    this might lead to longer convergence times. This increase increase in convergence time could be mitigated by using a lengthier rollout step size. As it would be more 
    ideal as the drone would not be gaining much information in smaller steps, and the large step size could help to achieve higher information gain. Thirdly,
    another key area for future research is implementing capabilites for tracking moving radiation sources, which wuld necessitate real-time belief updates, 
    adaptive path planning, predictive source movement modeling, and enhanced convergence strategies for mobile targets.
\end{document}
